%!TEX root = /Users/stevenmartell1/Documents/IPHC/SizeLimitz/docs/manuscript/SizeLimitz.tex


\section*{Discussion} % (fold)
\label{sec:discussion}

It is intuitive to think that imposing a maximum size limit would afford protection to sexually mature fish that grow beyond the size limit and that this would lead to an increase in spawning biomass (or reduce the level of depletion).  This does occur, but only if there is a very low discard mortality rate associated with releasing fish.  In the case examined here, with 140cm size limit and fishing at $F_{\rm{MSY}}$, the spawning biomass in each regulatory area remains nearly the same or  declines in comparison to no maximum size limit.  The reason for this decline is related to a discard mortality rate of 0.16 per year, and under MSY-based harvest policies, values of $F_{\rm{MSY}}$ would increase in areas where halibut grow to sufficient size and a maximum size limit is used in the harvest policy.

Shifts in the directed commercial selectivity schedule towards smaller halibut pose a conservation concern if the discard mortality rate is greater than 0, even if minimum size limits are in place. If individual IFQ holders are not accountable for their discard mortality of undersized fish, then fishing can continue until their quota is filled with legal-sized halibut.  If there is a shift towards catching smaller fish, or the probability of capturing a legal-size fish in a given area is low, then the corresponding increase in discard mortality  results in a higher overall total mortality rate that may not be accounted for if the shift in selectivity goes undetected.  This is of particular concern in assessment models that subtract off the under-sized fish in directed fisheries, where the estimates of under-size fish assumes selectivity in the commercial fishery has not changed (as is the current practice with the IPHC assessment model).  The reason for this concern, from a reference point perspective, is that changes in commercial selectivity are not updated in the assessment model because the model is fit only to composition data of legal fish and the selectivity curves are not updated.  Scenario 4 (shift in selectivity by -10cm) clearly demonstrated that estimates of $F_{\rm{MSY}}$ should decrease substantially if such a shift in commercial selectivity occurs.  Similar interactions have also been noted in recreational fisheries that use a combination of bag limits (akin to individual daily quotas) and size-limits \citep{goodyear1993spawning,coggins2007ecm}.

% section discussion (end)
